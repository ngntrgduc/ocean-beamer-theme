% change font size: https://texblog.org/2012/08/29/changing-the-font-size-in-latex/
\documentclass{beamer}
% \documentclass[aspectratio=169]{beamer} % For wider width
\usepackage[utf8]{inputenc}
\usepackage{tikz}
\usepackage{multicol}
\usepackage{xcolor}

% Coloring ---------------------------------------------------------------
\definecolor{ocean}{RGB}{4, 165, 229}           % Define main colors
\setbeamercolor{title}{fg=white, bg=ocean}      % title color
\setbeamercolor{frametitle}{fg=white, bg=ocean} % color of frame title

% Table of contents ---------------------------------------------------------------
\setbeamerfont{section number projected}{family=\rmfamily, series=\bfseries, size=\normalsize} % Font of bullets
\setbeamercolor{section number projected}{bg=ocean, fg=white} % Color of bullets
\setbeamercolor{section in toc}{fg=ocean}  % Sections color
\setbeamertemplate{section in toc}[square] % Add square bullet for sections

% Bullets style ---------------------------------------------------------------
% Item  bullets style
\setbeamertemplate{items}{$\blacksquare$} % square symbol for items
\setbeamercolor{itemize item}{fg=ocean}
\setbeamertemplate{itemize subitem}{\raisebox{1pt}{$\blacktriangleright$}} % triangle for subitems
\setbeamercolor{itemize subitem}{fg=ocean}
\setbeamertemplate{itemize subsubitem}{\raisebox{1pt}{\textbullet}} % bullet for subsubitems
\setbeamercolor{itemize subsubitem}{fg=ocean}

% Enumerate bullets style
\setbeamertemplate{enumerate item}{
  \usebeamercolor{item projected}
  \raisebox{1pt}{\colorbox{ocean}{\color{white}\footnotesize\insertenumlabel}}
}
\setbeamertemplate{enumerate subitem}{
  \usebeamercolor{item projected}
  \raisebox{0.75pt}{\colorbox{ocean}{\color{white}\scriptsize\insertenumlabel.\insertsubenumlabel}}
}
\setbeamertemplate{enumerate subsubitem}{
  \usebeamercolor{item projected}
  \raisebox{0.5pt}{\colorbox{ocean}{\color{white}\scriptsize\insertenumlabel.\insertsubenumlabel.\insertsubsubenumlabel}}
}

% Block customization ---------------------------------------------------------------
% \setbeamertemplate{blocks}[rounded] % Set rounded for block
\setbeamercolor{block title}{fg=white, bg=ocean}
\setbeamercolor{block body}{bg=ocean!10}
\setbeamercolor{block title alerted}{fg=white, bg=magenta}
\setbeamercolor{block body alerted}{bg=magenta!10}
\setbeamercolor{block title example}{fg=white, bg=gray}
\setbeamercolor{block body example}{bg=gray!10}

% Footer customization ---------------------------------------------------------------
% For more footer customization, visit https://github.com/ngntrgduc/ocean-beamer-theme/wiki#addchange-footer
\setbeamertemplate{footline}[frame number]              % Add page numbers in each slide
\setbeamercolor{page number in head/foot}{fg=ocean}     % Change color of page numbers
% \setbeamerfont{page number in head/foot}{size=\scriptsize}   % Make page numbers bigger
\setbeamertemplate{navigation symbols}{}                % Make navigations pane disappear

% Preamble ---------------------------------------------------------------
\title[Ocean Beamer Theme]{\huge Ocean Beamer Theme}
\author[ngntrgduc]{\large Trung-Duc Nguyen}
\institute[HCMUS]{
  \inst{}
  \large University of Science - VNUHCM \\Faculty of Mathematics and Computer Science\\
}
\date{\today}

% Insert logo ---------------------------------------------------------------
% 
% At the bottom right :
% \titlegraphic{
%     \begin{tikzpicture}[overlay,remember picture]
%         \node[left=0cm] at (current page.-30){
%             \includegraphics[height=2cm]{logo.png}
%         };
%     \end{tikzpicture}
% }
%
% Or at center :
% \titlegraphic{
%   \includegraphics[height=2.5cm]{logo.png}
% }


% Begin presentation ---------------------------------------------------------------
\begin{document}

% Creates the title slide.
\begin{frame}[plain] % [Plain] use to hide headline and footline of the first slide
    \maketitle 
\end{frame}


% Table of Contents slide -------------------------------------------------------
% Comment this to hide table of contents (with section) every section
\AtBeginSection[]{
    \begin{frame}<beamer>
        \frametitle{Table of Contents}
        \begin{columns}[T]
            \begin{column}{.5\textwidth}
                \tableofcontents[currentsection, subsectionstyle=hide, sections=1-4]
            \end{column}
            \begin{column}{.5\textwidth}
                \tableofcontents[currentsection, subsectionstyle=hide, sections=5-8]
            \end{column}
        \end{columns}
        % \tableofcontents[currentsection, subsectionstyle=hide]
    \end{frame}
}

\begin{frame}
    \frametitle{Table of Contents}
    \begin{columns}[T]
        \begin{column}{.5\textwidth}
            \tableofcontents[subsectionstyle=hide, sections={1-4}]
        \end{column}
        \begin{column}{.5\textwidth}
            \tableofcontents[subsectionstyle=hide, sections={5-8}]
        \end{column}
    \end{columns}
    % \tableofcontents[currentsection, subsectionstyle=hide]
\end{frame}


%----------------------------------------------------------------

\section{First section}
\subsection{Enumerate}
\begin{frame}
    \frametitle{Enumerate}
    \begin{enumerate}
        \item Sample 1.
        \item Sample 2.
            \begin{enumerate}
                \item Sample 1.
                    \begin{enumerate}
                        \item Sample 1.
                        \item Sample 2.
                    \end{enumerate}
                \item Sample 2.
            \end{enumerate}
    \end{enumerate}
\end{frame}


\subsection{Itemize}
\begin{frame}{Itemize}
    \begin{itemize}
        \item Item 1.
        \item Item 2.
            \begin{itemize}
                \item Subitem 1.
                    \begin{itemize}
                        \item Subsubitem 1.
                        \item Subsubitem 2.
                    \end{itemize}
                \item Subitem 2.
            \end{itemize}
    \end{itemize}
\end{frame}


\section{Second section}
\subsection{Sample block}
\begin{frame}
    \frametitle{Sample block}
    \begin{block}{Remark}
        Sample text in box.
    \end{block} 
        
    \begin{alertblock}{Important theorem}
        Sample text in box.
    \end{alertblock}
        
    \begin{examples}
        Sample text in box.
    \end{examples}
\end{frame}
    
\subsection{Two-column slide}
\begin{frame}
    \frametitle{Two-column slide}
    \begin{columns}
    \column{0.5\textwidth}
    This is a text in first column. \[\mbox{Some math : }y=f(x)\]
        \begin{itemize}
            \item First item in first column.
            \item Second item in first column.
        \end{itemize}
    
    \column{0.5\textwidth}
    This is a text in second column. \\
    This is another text in second column. 
    \end{columns}
\end{frame}


\section{Third section}
\subsection{First subsection}
\begin{frame}{First subsection}
    \alert{This text highlighted} because it's \color{red}{important.}
\end{frame}

\section{Fourth section}
\subsection{First subsection}
\begin{frame}{First subsection}
    This is ...
\end{frame}


\section{Fifth section}
\subsection{First subsection}
\begin{frame}{First subsection}
    This is a text.
\end{frame}


\subsection{Second subsection}
\begin{frame}{Second subsection}
    This is another text.
\end{frame}


\section{Conclusion}
\begin{frame}{Conclusion}
    \begin{itemize}
        \item Conclusion 1
        \item Conclusion 2
    \end{itemize}
\end{frame}

\begin{frame}[plain]
    \begin{tikzpicture}[overlay, remember picture]
        \node[anchor=center] at (current page.center) {
            \Huge \textcolor{ocean}{Q \& A}
        };
    \end{tikzpicture}
\end{frame}


\begin{frame}[plain]
    \begin{tikzpicture}[overlay, remember picture]
        \node[anchor=center] at (current page.center) {
            \Huge \textcolor{ocean}{Thank You}
        };
    \end{tikzpicture}
\end{frame}
 

\begin{frame}{References}
    \begin{itemize}
        \item References 1
        \item References 2
    \end{itemize}
\end{frame}


\end{document}