\documentclass{beamer}
\usepackage[utf8]{inputenc}
\usepackage{tikz}
\usepackage{multicol}

\usetheme{Antibes}

% Add page numbers in each slide
\setbeamertemplate{footline}[frame number] 

% Make navigations pane disappear
\setbeamertemplate{navigation symbols}{} 

% Define custom colors
\definecolor{MyCyan1}{RGB}{0, 170, 170}
\definecolor{MyCyan2}{RGB}{0, 221, 221}
\definecolor{MyCyan3}{RGB}{68, 255, 255}

% Change color of bullets in table of contents
\setbeamerfont{section number projected}{
  family=\rmfamily,series=\bfseries,size=\normalsize}
\setbeamercolor{section number projected}{bg=MyCyan3,fg=white}

\setbeamerfont{subsection number projected}{
  family=\rmfamily,series=\bfseries,size=\normalsize}
\setbeamercolor{subsection number projected}{bg=MyCyan3,fg=white}

\setbeamerfont{subsubsection number projected}{
  family=\rmfamily,series=\bfseries,size=\normalsize}
\setbeamercolor{subsubsection number projected}{bg=MyCyan3,fg=white}

% Change section color in table of contents
\setbeamercolor{section in toc}{fg=MyCyan3}

% Change item  bullets style
\setbeamercolor{itemize item}{fg=MyCyan3}
\setbeamertemplate{itemize subitem}{$\blacktriangleright$}
\setbeamercolor{itemize subitem}{fg=MyCyan3}
\setbeamertemplate{itemize subsubitem}{\textbullet}
\setbeamercolor{itemize subsubitem}{fg=MyCyan3}


% Change enumerate bullets style
\setbeamertemplate{enumerate item}{
  \usebeamercolor{item projected}
  \raisebox{0.8pt}{\colorbox{MyCyan3}{\color{white}\footnotesize\insertenumlabel}}
}

\setbeamertemplate{enumerate subitem}{
  \usebeamercolor{item projected}
  \raisebox{0.8pt}{\colorbox{MyCyan3}{\color{white}\footnotesize\insertenumlabel.\insertsubenumlabel}}
}

\setbeamertemplate{enumerate subsubitem}{
  \usebeamercolor{item projected}
  \raisebox{0.8pt}{\colorbox{MyCyan3}{\color{white}\footnotesize\insertenumlabel.\insertsubenumlabel.\insertsubsubenumlabel}}
}

% Change standard block colors
\setbeamercolor{block title}{bg=blue, fg=white}
\setbeamercolor{block body}{bg=blue!10}

% Change alert block colors
\setbeamercolor{block title alerted}{fg=white, bg=red}
\setbeamercolor{block body alerted}{bg=red!10}

% Change example block colors
\setbeamercolor{block title example}{fg=white, bg=green}
\setbeamercolor{block body example}{bg=green!10}

% Change heading colors
\setbeamercolor{title in head/foot}{fg=white, bg=black}
\setbeamercolor{section in head/foot}{fg=white, bg=MyCyan1}
\setbeamercolor{subsection in head/foot}{fg=white, bg=MyCyan2}
\setbeamercolor{title}{fg=white, bg=MyCyan3}
\setbeamercolor{frametitle}{fg=white, bg=MyCyan3}

% Add Author Name, Short Title, Date, and Pages to Footline
\makeatletter
\setbeamertemplate{footline}
{
\leavevmode%
\hbox{%
    \begin{beamercolorbox}[wd=.5\paperwidth,ht=2.25ex,dp=1ex,center]{section in head/foot}%
        \usebeamerfont{author in head/foot}\insertshortauthor~~\beamer@ifempty{\insertshortinstitute}{}{(\insertshortinstitute)} \hspace*{10em}
    \end{beamercolorbox}%
  
    \begin{beamercolorbox}[wd=.5\paperwidth,ht=2.25ex,dp=1ex,right]{section in head/foot}%
        \usebeamerfont{date in head/foot}\insertshortdate{}\hspace*{2em}
        \insertframenumber{} / \inserttotalframenumber\hspace*{2ex}
    \end{beamercolorbox}}%
\vskip0pt%
}
\makeatother

% Preamble ----------------------------------------------------------------------------------------------------------------------------------------------------------------------
\title[Ocean Beamer Theme]{\huge Ocean Beamer Theme}
\author[ngntrgduc]{\large Nguyen Trung Duc}
\institute[HCMUS]{
  \inst{}
  \large VNUHCM - University of Science \\Faculty of Mathematics and Computer Science\\
}
\date{\today}

% Insert logo -------------------------------------------------------------------------------------------------------------------------------------------------------------------
% 
% At the bottom right :
% \titlegraphic{
% \begin{tikzpicture}[overlay,remember picture]
% \node[left=0cm] at (current page.-30){
%     \includegraphics[height=2cm]{logo.png}
% };
% \end{tikzpicture}
% }
%
% Or at center :
% \titlegraphic{
% \includegraphics[height=2.5cm]{logo.png}
% }


% Begin presentation ------------------------------------------------------------------------------------------------------------------------------------------------------------
\begin{document}

%Creates the title slide.
\begin{frame}[plain] % [Plain] use to hide headline and footline of the first slide
    \maketitle 
\end{frame}


% Table of Contents slide -------------------------------------------------------------------------------------------------------------------------------------------------------
\AtBeginSection[]{
    \begin{frame}<beamer>
        \frametitle{Table of Contents}
        \begin{columns}[T]
            \begin{column}{.5\textwidth}
                \tableofcontents[currentsection,sections=1-4]
            \end{column}
            \begin{column}{.5\textwidth}
                \tableofcontents[currentsection,sections=5-8]
            \end{column}
        \end{columns}
    \end{frame}
}

\begin{frame}
    \frametitle{Table of Contents}
    \begin{columns}[t]
        \begin{column}{.5\textwidth}
            \tableofcontents[sections={1-4}]
        \end{column}
        \begin{column}{.5\textwidth}
            \tableofcontents[sections={5-8}]
        \end{column}
    \end{columns}
\end{frame}


%--------------------------------------------------------------------------------------------------------------------------------------------------------------------------------
\section{First section}
\subsection{Enumerate}
\begin{frame}
    \frametitle{Enumerate}
    \begin{enumerate}
        \item Sample 1.
        \item Sample 2.
            \begin{enumerate}
                \item Sample 1.
                    \begin{enumerate}
                        \item Sample 1.
                        \item Sample 2.
                    \end{enumerate}
                \item Sample 2.
            \end{enumerate}
    \end{enumerate}
\end{frame}


\subsection{Itemize}
\begin{frame}{Itemize}
    \begin{itemize}
        \item Item 1.
        \item Item 2.
            \begin{itemize}
                \item Subitem 1.
                    \begin{itemize}
                        \item Subsubitem 1.
                        \item Subsubitem 2.
                    \end{itemize}
                \item Subitem 2.
            \end{itemize}
    \end{itemize}
\end{frame}


\section{Second section}
\subsection{Sample block}
\begin{frame}
    \frametitle{Sample block}
    \begin{block}{Remark}
        Sample text in box.
    \end{block} 
        
    \begin{alertblock}{Important theorem}
        Sample text in box.
    \end{alertblock}
        
    \begin{examples}
        Sample text in box.
    \end{examples}
\end{frame}
    
\subsection{Two-column slide}
\begin{frame}
    \frametitle{Two-column slide}
    \begin{columns}
    \column{0.5\textwidth}
    This is a text in first column. \[\mbox{Some math : }y=f(x)\]
        \begin{itemize}
            \item First item in first column.
            \item Second item in first column.
        \end{itemize}
    
    \column{0.5\textwidth}
    This is a text in second column. \\
    This is another text in second column. 
    \end{columns}
\end{frame}




\section{Third section}
\subsection{First subsection}
\begin{frame}{First subsection}
    \alert{This text highlighted} because it's \colorbox{MyCyan3}{important.}
\end{frame}

\section{Fourth section}
\subsection{First subsection}
\begin{frame}{First subsection}
    Nothing here.
\end{frame}


\section{Fifth section}
\subsection{First subsection}
\begin{frame}{First subsection}
    This is a text.
\end{frame}


\subsection{Second subsection}
\begin{frame}{Second subsection}
    This is another text.
\end{frame}


\section{Conclusion}
\begin{frame}{Conclusion}
    \begin{itemize}
        \item Conclusion 1
        \item Conclusion 2
    \end{itemize}
\end{frame}


\begin{frame}{Q\&A}
    \begin{center}
        \Huge \textcolor{MyCyan3}{Thank You}
    \end{center}
\end{frame}
 

\begin{frame}{References}
    \begin{itemize}
        \item References 1
        \item References 2
    \end{itemize}
\end{frame}

\end{document}
